\documentclass[serif,mathserif]{article}
\usepackage[utf8x]{inputenc}
\usepackage{comment}
\usepackage[portuges]{babel}
\usepackage{booktabs}
\usepackage{listings}
\usepackage{dcolumn}
\usepackage{hyperref}
\usepackage{courier}
\usepackage{caption}

\lstdefinestyle{xml}{language=xml,tabsize=3,frame=lines,keywordstyle=\color{blue},commentstyle=\color{OliveGreen},stringstyle=\color{red},breaklines=true,showstringspaces=false,basicstyle=\footnotesize,emph={label}}
\lstdefinestyle{java}{language=java,tabsize=3,frame=lines,keywordstyle=\color{blue},commentstyle=\color{OliveGreen},stringstyle=\color{red},breaklines=true,showstringspaces=false,basicstyle=\footnotesize,emph={label}}
\DeclareCaptionFont{white}{\color{white}}
\DeclareCaptionFormat{listing}{\colorbox{gray}{\parbox{\textwidth}{#1#2#3}}}
\captionsetup[lstlisting]{format=listing,labelfont=white,textfont=white}
\renewcommand{\lstlistingname}{Código}

\lstloadlanguages{Ruby}
\lstdefinelanguage{Smalltalk}{ 
  morekeywords={true,false,self,super,nil}, 
  sensitive=true, 
  morecomment=[s]{"}{"}, 
  morestring=[d]', 
  style=SmalltalkStyle 
} 
\lstdefinelanguage{JavaScript}{
  keywords={break, case, catch, continue, debugger, default, delete, do, else, finally, for, function, if, in, instanceof, new, return, switch, this, throw, try, typeof, var, void, while, with},
  morecomment=[l]{//},
  morecomment=[s]{/*}{*/},
  morestring=[b]',
  morestring=[b]",
  sensitive=true
}
\lstdefinestyle{SmalltalkStyle}{ 
  literate={:=}{{$\gets$}}1{^}{{$\uparrow$}}1 
}

\author{ 
    \\ Rodrigo di Lorenzo Lopes \\  \texttt{rodrigo.lorenzo@abril.com.br}
}
\title{Curso: Ruby on Rails}

\begin{document}
\maketitle
 
\tableofcontents

\section{Introduction}

\subsection{\#\#O que sabemos}

Ruby é uma linguagem dinâmica que favorece a metaprogramação.

A linguagem favorece a criação de artefatos que tornam a programação mais enxuta e expressiva.

\subsection{\#\#Referências}

\href{http://guias.rubyonrails.com.br/}{Ruby on Rails guides}

\section{\#\#Ecosistema Ruby on Rails}

\subsection{\#\#Gems}

Sistema de pacotes do Ruby

\begin{itemize}
  \item Instalação do pacote (similar ao apt-get)
  \item Repositório de pacotes 
\end{itemize}

\subsubsection{\#\#Exemplo de uso}
\begin{lstlisting}
gem install methodize
\end{lstlisting}

\begin{verbatim}
Successfully installed methodize-0.2.2
1 gem installed
\end{verbatim}

\subsection{\#\#Bundler}

\subsubsection{\#\#Bundler by Bundler}

Bundler tracks an application's code and the rubygems it needs to run, so that an application will always have the exact gems (and versions) that it needs to run.

\subsubsection{\#\#Bundler + Gemfile}

Assim como o apt-get, gem install irá procurar a última versão da biblioteca procurada.

Um Gemfile é similar ao pom.xml do Maven

\begin{lstlisting}
# These gems are in the :default group
gem "nokogiri"
gem "sinatra"

gem "wirble", :group => :development

group :test do
  gem "rspec"
  gem "faker"
end
\end{lstlisting}

\subsubsection {\#\#bundle install}

Para gerar uma aplicação
\begin{verbatim}
bundle install
\end{verbatim}

\subsection {\#\#DSL}
Gemfile é bom exemplo como Ruby favorece a criação de DSLs.

Ruby possui três comandos para interpretação de texto:
\begin{enumerate}
  \item eval
  \item instance\_eval
  \item class\_eval
\end{enumerate}

\subsection{\#\#class\_eval}
\begin{lstlisting}[language=ruby]
mod.class_eval(string [, filename [, lineno]]) => obj
\end{lstlisting}


\begin{quote}
Evaluates the string or block in the context of mod. This can be used to add methods to a class.
\end{quote}


\subsubsection{\#\#instance\_eval}
\begin{lstlisting}[language=ruby]
obj.instance_eval {| | block } => obj
\end{lstlisting}

\begin{quote}
Evaluates a string containing Ruby source code, or the given block, within the context of the receiver (obj). In order to set the context, the variable self is set to obj while the code is executing, giving the code access to obj’s instance variables.
\end{quote}

\subsubsection{\#\#Mind trick - class\_eval vs instance\_eval}

\begin{lstlisting}[language=ruby]
Foo = Class.new
Foo.class_eval do
  def class_bar
    "class_bar"
  end
end
Foo.instance_eval do
  def instance_bar
    "instance_bar"
  end
end
\end{lstlisting}

\subsubsection{\#\#What's going to happen? - class\_eval vs instance\_eval}
\begin{lstlisting}[language=ruby]
Foo.class_bar
Foo.new.class_bar 
Foo.instance_bar
Foo.new.instance_bar
\end{lstlisting}


\subsubsection{\#\#What's going to happen? - class\_eval vs instance\_eval}
\begin{lstlisting}[language=ruby]
Foo.class_bar   #=> undefined method 'class_bar' for Foo:Class
Foo.new.class_bar #=> "class_bar"
Foo.instance_bar #=> "instance_bar"
Foo.new.instance_bar #=> undefined method 'instance_bar' for
\end{lstlisting}

\subsubsection{\#\#É tudo sobre o self - class\_eval vs instance\_eval}
\begin{tabular}{ l | c | c |  r }
mechanism &	method resolution &	method definition &	new scope? \\
class Person &	Person &	same &	yes \\
class << Person	Person's & metaclass &	same &	yes\\
Person.class\_eval & Person & same & no\\
Person.instance\_eval & Person & Person's metaclass & no
\end{tabular}


\subsection {\#\#Desafio - DSL - Receita Macarronada com Queijo}

\href{http://rubylearning.com/blog/2010/11/30/how-do-i-build-dsls-with-yield-and-instance_eval/}{how-do-i-build-dsls}

\subsubsection{\#\#Qual a estrutura deve ser criada para cuidar disso?}
\begin{lstlisting}[language=ruby]
mac_and_cheese = Recipe.new("Mac and Cheese") do
  ingredient "Water", :amount => "2 cups"
  ingredient "Noodles", :amount => "1 cup"
  ingredient "Cheese", :amount => "1/2 cup"

  step "Heat water to boiling.", :for => "5 minutes"
  step "Add noodles to boiling water.", :for => "6 minutes"
  step "Drain water."
  step "Mix cheese in with noodles."
end
\end{lstlisting}

\subsubsection{\#\#Que tal assim?}
\begin{lstlisting}[language=ruby]
def initialize(name, &block)
  self.name = name
  self.ingredients = []
  self.instructions = []
  instance_eval &block
end
\end{lstlisting}

\subsubsection{\#\#Métodos auxiliares}
\begin{lstlisting}[language=ruby]
  def ingredient(name, options = {})
    ingredient = name
    ingredient << " (#{options[:amount]})" if options[:amount]
    ingredients << ingredient
  end

  def step(text, options = {})
    instruction = text
    instruction << " (#{options[:for]})" if options[:for]
    instructions << instruction
  end
\end{lstlisting}

\subsubsection{\#\#DSL - Receita - Resultado}
\begin{lstlisting}[language=ruby]
file = File.open(ARGV[0], "rb")
contents = file.read
recipe = Recipe.new ARGV[0], contents
p recipe

ruby recipes.rb mac_cheese.txt 
#<Recipe:0x00000101086f90 @name="mac_cheese.txt", 
@ingredients=["Water (2 cups)", 
#"Noodles (1 cup)", "Cheese (1/2 cup)"], 
#@instructions=["Heat water to boiling. (5 minutes)", 
#"Add noodles to boiling water. (6 minutes)", 
#"Drain water.", "Mix cheese in with noodles."]>
\end{lstlisting}


\subsection{\#\#Gemfile detalhes}

\subsubsection{Declarando Gem servers}
\begin{lstlisting}[language=ruby]
source :rubygems
source "http://rubygems.org"
source :rubyforge
source "http://gems.rubyforge.org"
source :gemcutter
source "http://gemcutter.org"
\end{lstlisting}

\subsubsection{\#\#Especificando a versão adequada das gemas}
\begin{lstlisting}[language=ruby]
gem "nokogiri"
gem "rails", "3.0.0.beta3"
gem "rack",  ">=1.0"
gem "thin",  "~>1.1"
\end{lstlisting}

\subsubsection{\#\#Nota sobre os Espeficiadores de versão}
Most of the version specifiers, like $\geq 1.0$, are self-explanatory. The specifier \verb|~>| has a special meaning, best shown by example. \verb|~> 2.0.3| is identical to $\geq 2.0.3$ and $<  2.1$. \verb|~> 2.1| is identical to $ \geq 2.1$ and $< 3.0$.

\subsubsection{\#\#Obtendo gemas do Git}
\begin{lstlisting}[language=ruby]
gem "nokogiri", :git => "git://github.com/tenderlove/nokogiri.git", :branch => "1.4"

git "git://github.com/wycats/thor.git", :tag => "v0.13.4"
gem "thor"
\end{lstlisting}

\subsubsection{\#\#Grupos no Gemfile}
É possível atribuir gemas que são úteis em contextos específicos
\begin{lstlisting}[language=ruby]
# These gems are in the :default group
gem "nokogiri"
gem "sinatra"

gem "wirble", :group => :development

group :test do
  gem "rspec"
  gem "faker"
end
\end{lstlisting}
  
Install all gems, except those in the listed groups. Gems in at least one non-excluded group will still be installed.
\begin{verbatim}
$ bundle install --without test development
\end{verbatim}

\subsection{\#\#RVM}

\subsubsection{\#\#Vantagens}

\begin{enumerate}
  \item Permite a instalação de várias versões do Ruby
  \item Permite a criação de Gemsets que agrupam versões específicas de gemas para certas funcionalidades.
\end{enumerate}


\subsubsection{\#\#Instalação}

\begin{verbatim}
    $ \curl -L https://get.rvm.io | bash -s stable --ruby
\end{verbatim}

\subsubsection{\#\#Instalar um novo Ruby}

\begin{verbatim}
    $ rvm install jruby
    $ rvm get stable
\end{verbatim}

Para listar as versoes instaladas:
\begin{verbatim}
    $ rvm list
\end{verbatim}

Para listar a versão atual:
\begin{verbatim}
    $ rvm current
\end{verbatim}


\subsubsection{\#\#Criando Gemsets}

\begin{verbatim}
    $ rvm gemset create
\end{verbatim}

Para listar os gemset criados para a versão de ruby atual:
\begin{verbatim}
    $ rvm gemset list
\end{verbatim}

Para usar um gemset específico:
\begin{verbatim}
    $ rvm use ruby_version@gemset
\end{verbatim}

\subsubsection{\#\#Criando Gemsets II}
Na prática

\begin{enumerate}
  \item Criar um gemset para o projeto
  \item Criar um arquivo .rvm no projeto:
\begin{verbatim}
    echo "rvm user gemset xxx" > .rvmrc
\end{verbatim}
 \item Voltar ao diretório
 \item Usar bundler normalmente
\end{enumerate}

\subsection{\#\#RSpec}

\url{http://rspec.info/}

RSpec is testing tool for the Ruby programming language. Born under the banner of Behaviour-Driven Development.

\subsubsection{\#\#Get Started Now}

\begin{verbatim}
$ gem install rspec
\end{verbatim}

\subsubsection{\#\#Rspec - Simple Example}
\begin{lstlisting}[language=ruby]
# bowling_spec.rb
require 'bowling'

describe Bowling, "#score" do
  it "returns 0 for all gutter game" do
    bowling = Bowling.new
    20.times { bowling.hit(0) }
    bowling.score.should eq(0)
  end
end
\end{lstlisting}

\subsubsection{\#\#Rspec - Running}
\begin{verbatim}
 rspec bowling_spec.rb

./bowling_spec.rb:4:
  uninitialized constant Bowling
\end{verbatim}


\subsubsection{\#\#Rspec - coding BDD}
\begin{lstlisting}[language=ruby]
# bowling.rb
class Bowling
  def hit(pins)
  end

  def score
    0
  end
end
\end{lstlisting}

\subsubsection{\#\#Rspec - Running Ok!}
\begin{verbatim}
$ rspec bowling_spec.rb --format nested

Bowling#score
  returns 0 for all gutter game

Finished in 0.007534 seconds

1 example, 0 failures
\end{verbatim}

\subsubsection{\#\#Rspec - Basic structure}
\begin{verbatim}
"Describe an order."
"It sums the prices of its line items."
\end{verbatim}

\begin{lstlisting}[language=ruby]
describe Order do
  it "sums the prices of its line items" do
    order = Order.new
    order.add_entry(LineItem.new(:item => Item.new(
      :price => Money.new(1.11, :USD)
    )))
    order.add_entry(LineItem.new(:item => Item.new(
      :price => Money.new(2.22, :USD),
      :quantity => 2
    )))
    expect(order.total).to eq(Money.new(5.55, :USD))
  end
end
\end{lstlisting}

\subsubsection{\#\#Rspec - Nested Groups}
\begin{lstlisting}[language=ruby]
 describe Order do
  context "with no items" do
    it "behaves one way" do
      # ...
    end
  end

  context "with one item" do
    it "behaves another way" do
      # ...
    end
  end
end
\end{lstlisting}

 
\subsection{\#\#Mongo}

\url{http://www.mongodb.org/}

Mongo é um banco de dados NoSQL.

\url{http://docs.mongodb.org/manual/tutorial/getting-started-with-the-mongo-shell}
\url{http://docs.mongodb.org/manual/reference/mongo-shell/}
\url{http://api.mongodb.org/js/2.3.2/symbols/_global_.html}
\url{http://docs.mongodb.org/manual/tutorial/getting-started/}


\subsubsection{\#\#Principais características}
\begin{itemize}
  \item Document-Oriented Storage
  \item Full Index Support
  \item Auto-Sharding
  \item Querying
  \item Map/Reduce
\end{itemize}

\subsubsection{\#\#Instalação}

Baixar e instalar o pacote adequato em \url{http://www.mongodb.org/downloads}.

Executar o daemon: mongod.

\subsubsection{\#\#Mongo Shell}

Mongo possui uma inteface interessante em shell baseado em javascript.

\begin{lstlisting}[language=Javascript]
 for (var i=0; i<100; i++) {
   print(i);
 }
\end{lstlisting}

\subsubsection{\#\#Mongo Shell - Database}

\begin{lstlisting}[language=Javascript]
show dbs // lista as bases
use nome_da_base // cria e/ou usa uma base
db // mostra a base atual selecionada
show collections // mostra as colecoes da base
db["3test"].find() // busca todos documentos dessa colecao
db.getCollection("3test").find() // isso faz a mesma coisa
\end{lstlisting}

\subsubsection{\#\#Mongo Shell - Print}
\begin{lstlisting}[language=Javascript]
db.<collection>.find().pretty()
\end{lstlisting}

In addition, you can use the following explicit print methods in the mongo shell:
\begin{lstlisting}[language=Javascript]
print() // to print without formatting
print(tojson(<obj>)) // to print with JSON 
            //  formatting and equivalent to printjson()
printjson() // to print with JSON formatting
            // and equivalent to print(tojson(<obj>))
\end{lstlisting}


\subsubsection{Mongo Profile - Exemplo}
\begin{lstlisting}[language=Javascript]

\end{lstlisting}


\begin{lstlisting}[language=Javascript]
db.setProfilingLevel(0)
\end{lstlisting}



\subsection{\#\#MongoMapper}



\section{\#\#Rails - Primeiros Passos}

\subsection{\#\#Rails HelloWorld}

\section{\#\#Model}

\subsection{\#\#Rails Database Migrations}
This guide covers how you can use Active Record migrations to alter your database in a structured and organized manner.

\subsubsection{\#\#Rails Database Migrations in Mongo}

\subsubsection{\#\#Rails Database Migrations - Take that}
Não há necessidadde de migrations porque estamos usando mongo.
Evitamos dor e sofrimento.

\subsection{\#\#MongoMapper Validations and Callbacks}
This guide covers how you can use Active Record validations and callbacks.

\subsection{\#\#MongoMapper Associations}
This guide covers all the associations provided by Active Record.


\subsubsection{\#\#Query Interface}
This guide covers the database query interface provided by Active Record.



\section{\#\#Views}

\subsection{\#\#Layouts e Renderização no Rails}
  Este guia cobre o básico dos recursos de layout do Action Controller e Action View,
  incluindo renderização e redirecionamento, usando blocos content\_for, e trabalhando com partials.

\subsection{\#\#Action View Form Helpers}
  Guia para utilização dos Form helpers do Rails.

\subsection{\#\#Jsonify}

\subsection{\#\#REST in rails}



Controllers
Visão Geral do Action Controller
Este guia explica como os controllers trabalham e como eles se encaixam no ciclo de requisição de sua aplicação. Ele inclue sessões, filtros, cookies, streaming de dados, como trabalhar com exceções levantadas por um request, entre outros tópicos.

Roteamento Rails de Fora para Dentro
Este guia cobre as maravilhas do sistema de roteamento do Rails. Se você quer entender como usar as rotas em suas aplicações Rails, comece por aqui.

Rails on Rack
Este guia cobre a integração do Rails com Rack e sua interface com componentes Rack.




\end{document}
